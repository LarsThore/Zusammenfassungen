\documentclass[english]{scrartcl}
 
% ++++++++++++++++++++++++++++  PACKAGES  ++++++++++++++++++++++++++++++++++++++++ 
 
\usepackage[utf8]{inputenc}
\usepackage[T1]{fontenc}
%\usepackage{lmodern}
\usepackage[english]{babel}
\usepackage{amsmath}
\usepackage{amssymb}
\usepackage{graphicx}
%\usepackage{multicol}
%\usepackage[printonlyused]{acronym}
%\usepackage{framed}
\usepackage{color}
\usepackage{gensymb}
\usepackage{csquotes}

\setcounter{MaxMatrixCols}{15}
\setlength{\parindent}{0pt}				% sets indentation space to 0

% ++++++++++++++++++++ NEW COMMANDS AND ENVIRONMENTS ++++++++++++++++++++++++

\renewcommand{\vec}[1]{\boldsymbol{#1}}
\newcommand{\minus}{{\;} - {\;}}
\newcommand{\plus}{{\;} + {\;}}
\newcommand{\tr}{\text{tr}}
\newcommand{\dev}{\text{dev}}

\newenvironment{myitemize}{ \begin{itemize}
		\setlength{\itemsep}{0pt}
		\setlength{\parskip}{0pt}
		\setlength{\parsep}{0pt}     }
	{ \end{itemize}                  }

% ++++++++++++++++++++++++++++ BIBLIOGRAPHY +++++++++++++++++++++++++++++++++

\bibliographystyle{unsrt}%{backend=biber,style=numeric]{biblatex}

% +++++++++++++++++++++++ Title and Information +++++++++++++++++++++++++++++

\title{Inhomogeneous Material Properties in Solid Mechanical Systems}
\date{Thore Bergmann \\
	Master Studies 2015-2017 \\
	$ {\;} $ \\
	\includegraphics*[height = 3cm]{logo} \\
	$ {\;} $ \\
	Advisor: Dr. Stefan Sandfeld}

\begin{document}	

\maketitle
\clearpage

\section*{Continuum Mechanical Foundations and Notations}


\subsubsection*{The von Mises Stress or \enquote{Equivalent Tensile Stress}}

How can one compare two stresses?
\begin{myitemize}
	\item Cauchy stress tensor with 9 compoenents
	\begin{equation}
	\sigma = \begin{bmatrix}
	\sigma_{11}	& \sigma_{12}	 & \sigma_{13}	 \\ 
	\sigma_{21}	& \sigma_{22}	 & \sigma_{23}	 \\ 
	\sigma_{31}	& \sigma_{32}	 & \sigma_{33}	
	\end{bmatrix} = \begin{bmatrix}
	\sigma_{x}	& \tau_{xy}	 & \tau_{xz}	 \\ 
	\tau_{yx}	& \sigma_{y}	 & \tau_{yz}	 \\ 
	\tau_{zx}	& \tau_{zy}	 & \sigma_{z}	
	\end{bmatrix} \nonumber
	\end{equation}
	
	\item The von Mises stress or equivalent tensile stress
	\begin{equation}
	\sigma^{2}_{v} = \dfrac{1}{2} \left [ (\sigma_{11} \minus \sigma_{22})^{2} + (\sigma_{11} \minus \sigma_{22})^{2} + (\sigma_{11} \minus \sigma_{22})^{2} \plus 6 ( \sigma^{2}_{23} + \sigma^{2}_{31} + \sigma^{2}_{12}) \right ] \nonumber
	\end{equation}
	
\end{myitemize}

What are the properties of the von Mises stress?
\begin{myitemize}
	\item scalar stress measure
	\item two stress states with equal distortion energy have equal von Mises stresses 
	\item used to formulate the von Mises yield criterion
\end{myitemize}

What is the von Mises yield criterion?

\begin{myitemize}
	\item criterion, which is used to predict yielding of materials
	\item begins when the second deviatoric stress inavriant $ J_{2} $ rechaes a critical value
	\begin{equation*}
	\sigma_{v} = \sigma_{y} = \sqrt{3J_{2}}
	\end{equation*}
	\item stress values larger than the yield stress cannot be reached
\end{myitemize}

In which space does the von Mises yield surface live in and how does the -- theoretical -- surface look like?

\begin{myitemize}
	\item in the principal stress space
	\item like a cylinder (with the rotational axis: $ \sigma_{x} =  \sigma_{y} =  \sigma_{z} $ )
\end{myitemize}

What are principal stresses?

\begin{myitemize}
	\item the stresses on the diagonal line of the Cauchy stress tensor
\end{myitemize}

What is the work conjugate counterpart of the equivalent stress and how is it defined?

\begin{myitemize}
	\item equivalent strain or von Mises strain $ \varepsilon_{eq} $
	\begin{equation*}
		\varepsilon_{eq} = \sqrt{\dfrac{2}{3} \vec{\varepsilon}^{\text{dev}} : \vec{\varepsilon}^{\text{dev}}}
	\end{equation*}
	with 
	\begin{equation*}
		\vec{\varepsilon}^{\text{dev}} = \vec{\varepsilon} \minus \dfrac{1}{3} \text{tr}(\vec{\varepsilon}) \vec{1}
	\end{equation*}
\end{myitemize}

What are the first and the second invariant of the strain tensor?
	\begin{equation*}
		J_{1} = \tr (\vec{\varepsilon}), \qquad \qquad J_{2} = \dfrac{1}{2} \sigma^{\text{dev}} : \sigma^{\dev}
	\end{equation*}
\begin{myitemize}
	\item the second invariant can also be defined in several other ways (e.g. in dependence of the strain tensor itself)
\end{myitemize}

What is the expression for the equivalent stress in terms of the deviatoric parts of the stress tensor?
\begin{equation*}
\sigma_{\text{eq}} = \sqrt{\dfrac{3}{2} \vec{\sigma}^{\text{\dev}} : \vec{\sigma}^{\text{\dev}}}
\end{equation*}

What are the plain strain conditions and why are they how they are?
\begin{myitemize}
	\item the length of a structure in one direction is much larger than in the other two directions
	\item strains associated with the larger direction are constrained by the nearby material and are small compared to the cross-sectional strains
\end{myitemize}

How do the strain and stress tensor lokk like in the plane strain approximation?
\begin{equation*}
\vec{\varepsilon} = \begin{bmatrix}
\varepsilon_{11}	& \varepsilon_{12}	 & 0	 \\ 
\varepsilon_{21}	& \varepsilon_{22}	 & 0	 \\ 
0	& 0	 & 0	
\end{bmatrix},  \qquad \qquad 
\vec{\sigma} = \begin{bmatrix}
\sigma_{11}	& \sigma_{12}	 & 0	 \\ 
\sigma_{21}	& \sigma_{22}	 & 0	 \\ 
0	& 0 & \sigma_{33}	
\end{bmatrix} 
\end{equation*}

What is elastic material behaviour?
\begin{myitemize}
	\item the path during loading is also followed during unloading
	\item reversible behaviour
\end{myitemize}


What is plastic material behaviour?
\begin{myitemize}
	\item residual irreversible deformation for the loading - unloading experiment
\end{myitemize}


What is ideal plastic behaviour?
\begin{myitemize}
	\item irreversible plastic deformation occurs when a threshold is reached (yield stress) 
	\item no hardening (stress stays constant for increasing strain)
	\item except for this the behaviour is ideally elastic 
\end{myitemize}

\textcolor{blue}{What is the yield condition for ideally plastic behaviour?}
\begin{equation*}
	f(\sigma, \sigma_{y}) \le 0
\end{equation*}

Of which term is the total strain made of?
\begin{myitemize}
	\item an elastic part and a plastic part
	\begin{equation*}
		\varepsilon = \varepsilon^{e}  \plus \varepsilon^{p}
	\end{equation*}
\end{myitemize}

How is hardening defined in the elastic-plastic framework?
\begin{myitemize}
	\item threshold evolves with loading history
\end{myitemize}

\textcolor{blue}{What is the difference between isotropic and kinematic hardening?} \\
\textbf{(Bilder)} \\

What is the relationship between effective plastic strain $ \bar{\varepsilon}^{p} $ and effective stress $ \bar{\sigma} $ under linear hardening behaviour?
\begin{equation*}
\Delta \bar{\varepsilon}^{p} = \dfrac{\bar{\sigma}(t) - \sigma_{y}}{E^{p}}, \qquad \qquad E^{p} = \dfrac{E \; E_{T}}{E \minus E_{T}}
\end{equation*}
\begin{myitemize}
	\item $ E^{T} $: hardening modulus \textbf{(Bilder)}
\end{myitemize}

What is necessary to describe the effective stress of isotropic linear hardening?
\begin{myitemize}
	\item the hardening modulus $ E^{T} $ \textbf{(Bilder)}
	\item $ \bar{\sigma} $ can be expressed as a function of $ E^{T} $ only 
\end{myitemize}

\textcolor{blue}{What is the difference between yield stress and flow stress?}


\section*{Averaging Elastic Material Properties with FEM}

\textcolor{blue}{Axel fragen ob das wichtig ist...} \\

\section*{Modelling of Shear Band Formation}

...see other summaries (e.g. Statistical Analysis in Materials Simulation)

\end{document}
