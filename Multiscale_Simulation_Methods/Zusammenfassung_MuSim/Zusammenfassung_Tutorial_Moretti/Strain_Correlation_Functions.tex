\documentclass[english]{scrartcl}
 
% ++++++++++++++++++++++++++++  PACKAGES  ++++++++++++++++++++++++++++++++++++++++ 
 
\usepackage[utf8]{inputenc}
\usepackage[T1]{fontenc}
%\usepackage{lmodern}
\usepackage[english]{babel}
\usepackage{amsmath}
\usepackage{amssymb}
\usepackage{graphicx}
%\usepackage{multicol}
%\usepackage[printonlyused]{acronym}
%\usepackage{framed}
\usepackage{color}
\usepackage{gensymb}
\usepackage{csquotes}

\setcounter{MaxMatrixCols}{15}
\setlength{\parindent}{0pt}				% sets indentation space to 0

% ++++++++++++++++++++ NEW COMMANDS AND ENVIRONMENTS ++++++++++++++++++++++++

\renewcommand{\vec}[1]{\boldsymbol{#1}}
\newcommand{\minus}{{\;} - {\;}}
\newcommand{\plus}{{\;} + {\;}}
\newcommand{\tr}{\text{tr}}
\newcommand{\dev}{\text{dev}}

\newenvironment{myitemize}{ \begin{itemize}
		\setlength{\itemsep}{0pt}
		\setlength{\parskip}{0pt}
		\setlength{\parsep}{0pt}     }
	{ \end{itemize}                  }

% ++++++++++++++++++++++++++++ BIBLIOGRAPHY +++++++++++++++++++++++++++++++++

\bibliographystyle{unsrt}%{backend=biber,style=numeric]{biblatex}

% +++++++++++++++++++++++ Title and Information +++++++++++++++++++++++++++++

\title{Statistical Analysis in Materials Simulation}
\date{Thore Bergmann \\
	Master Studies 2015-2017 \\
	$ {\;} $ \\
	\includegraphics*[height = 3cm]{logo} \\
	$ {\;} $ \\
	Advisor: Paolo Moretti}

\begin{document}	

\maketitle
\clearpage

\section*{Strain Correlation Functions}

When are strain correlation functions needed?
\begin{myitemize}
	\item analysing tensile tests of amorphous specimen
	\subitem - formation of localized shear bands
\end{myitemize}

What is the prerequisite for shear band formation?
\begin{myitemize}
	\item $ J_{2} $-plasticity
\end{myitemize}

How is the yield stress of the gaussian quadrature points in an FEM simulation of an amorphous material?
\begin{myitemize}
	\item randomly uniform distributed
\end{myitemize}

Why do shear bands form at 45$ ^{\circ} $-angle?
\begin{myitemize}
	\item along this direction the shear strain is maximum \textcolor{blue}{(is that true?)}
\end{myitemize}

What do the shear bands tell us?
\begin{myitemize}
	\item that the shear strain is maximum in 45$ ^{\circ} $-angle direction w.r.t. the loading direction
\end{myitemize}

What is the (spacial) strain correlation function?
\begin{myitemize}
	\item a quantitative tool that allows to detect correlations without the need for visual inspection of the simulation result
\end{myitemize}

How do you build the strain correlation function?
\begin{myitemize}
	\item by comparing the deviation of strain at two points from the average strain
	\begin{equation*}
		C(s_{x}, s_{y}) = \left < [\varepsilon(\vec{r}) - \varepsilon ] [\varepsilon(\vec{r+\vec{s}}) - \varepsilon ] \right >_{r}
	\end{equation*}
\end{myitemize}

How do you know from the strain correlation function plot that there is a correlation in a certain direction?
\begin{myitemize}
	\item the correlations are long ranged
\end{myitemize}

How is the algorithm for calculating the strain correlation function?
\begin{myitemize}
	\item choose an inner box of the specimen which you want to visualize (some distance to the boundary is needed for algorithmic convenience)
	\item begin with one element in the inner box and calculate the strain correlations for this point with all other points inside a certain distance (e.g. inside a square or circle)
	\item loop through all the points of the inner box
	\item take the average of the correlations for all the points of the inner box
\end{myitemize}

















\end{document}
